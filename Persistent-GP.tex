% Persistent-GP.tex
\begin{hcarentry}{Persistent} 
\label{persistent}
\report{Michael Snoyman}%05/13
\participants{Greg Weber, Felipe Lessa}
\status{stable} 
\makeheader

Persistent is a type-safe data store interface for Haskell.
Haskell has many different database bindings available. However, most
of these have little knowledge of a schema and therefore do not provide
useful static guarantees. Persistent is designed to work across different databases, and works on Sqlite, PostgreSQL, MongoDB, and MySQL, with an experimental backend for CouchDB.

We will soon be having a 1.2 release, mostly featuring a refactoring of the module hierarchy. We're taking this opporunity to clean up a few idiosyncracies in the API and make the documentation a bit more helpful, but otherwise the library is remaining unchanged.

In order to accomodate various different backend types, Persistent is broken up into multiple components (separated by type classes). There is one for storage/serialization, one for uniqueness, and one for querying. This means that anyone wanting to create database abstractions can re-use the battle-tested persistent storage/serialization layer. Persistent's query layer is universal across different backends and uses combinators:

\begin{code}
selectList [  PersonFirstName ==. "Simon", 
              PersonLastName ==. "Jones"] []
\end{code}

There are some drawbacks to the query layer: it doesn't cover every use case. Persistent has built-in some very good support for raw SQL. One can run arbitrary SQL queries and get back Haskell records or types for single columns. In addition, Felipe Lessa has created a very powerful library called esqueleto for creating more powerful SQL queries.

Persistent also has the ability to store embedded objects. One can store a list or a Map inside a column/field. The current implementation is most useful for MongoDB. In SQL an embedded object is stored as JSON.

\FuturePlans 
Future directions for Persistent:
\begin{compactitem}
\item Full CouchDB support
\item A MongoDB specific query layer
\item Adding key-value databases like Redis without a query layer.
\end{compactitem}

Most of Persistent development occurs within the Yesod~\cref{yesod} community.
However, there is nothing specific to Yesod about it.
You can have a type-safe, productive way to store data,
even on a project that has nothing to do with web development.

\FurtherReading 
\url{http://www.yesodweb.com/book/persistent} 
\url{http://hackage.haskell.org/package/esqueleto}
\end{hcarentry}
