% Conduit-MC.tex
\begin{hcarentry}[updated]{Conduit}
\label{conduit}
\report{Michael Snoyman}%05/13
\status{stable}
\makeheader

While lazy I/O has served the Haskell community well for many purposes in the past, it is not a panacea. The inherent non-determinism with regard to resource management can cause problems in such situations as file serving from a high traffic web server, where the bottleneck is the number of file descriptors available to a process.

Left fold enumerators have been the most common approach to dealing with streaming data without using lazy I/O. While it is certainly a workable solution, it requires a certain inversion of control to be applied to code. Additionally, many people have found the concept daunting. Most importantly for our purposes, certain kinds of operations, such as interleaving data sources and sinks, are prohibitively difficult under that model.

The conduit package was designed as an alternate approach to the same problem. The root of our simplification is removing one of the constraints in the enumerator approach. In order to guarantee proper resource finalization, the data source must always maintain the flow of execution in a program. This can lead to confusing code in many cases. In conduit, we separate out guaranteed resource finalization as its own component, namely the ResourceT transformer.

Once this transformation is in place, data producers, consumers, and transformers (known as Sources, Sinks, and Conduits, respectively) can each maintain control of their own execution, and pass off control via coroutines. The user need not deal directly with any of this low-level plumbing; a simple monadic interface (inspired greatly by the pipes package) is sufficient for almost all use cases.

Since its initial release, conduit has been through many design iterations, all the while keeping to its initial core principles. Since the last HCAR, we've released version 1.0. This release introduces a simplification of the public facing API, optimizing for the common use cases. This was a minor change, and the conduit ecosystem has already caught up. The package has been in a mature state for quite some time now, and can be relied upon for most streaming data needs.

There is a rich ecosystem of libraries available to be used with conduit, including cryptography, network communications, serialization, XML processing, and more. The Web Application Interface was the original motivator for creating the library, and continues to use it for expressing request and response bodies between servers and applications. As such, conduit is also a major player in the Yesod ecosystem.

The library is available on Hackage. There is an interactive tutorial available on the FP Complete School of Haskell. You can find many conduit-based packages in the Conduit category on Hackage as well.

\FurtherReading
\begin{compactitem}
\item \url{http://hackage.haskell.org/package/conduit}
\item \url{https://www.fpcomplete.com/user/snoyberg/library-documentation/conduit-overview}
\item \url{http://hackage.haskell.org/packages/archive/pkg-list.html#cat:conduit}
\end{compactitem}
\end{hcarentry}
