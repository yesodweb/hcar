% Yesod-GY.tex
\begin{hcarentry}{Yesod} 
\label{yesod}
\report{Michael Snoyman}%05/13
\participants{Greg Weber, Luite Stegeman, Felipe Lessa}
\status{stable} 
\makeheader

Yesod is a traditional MVC RESTful framework. By applying Haskell's strengths to this paradigm, Yesod helps users create highly scalable web applications.

Performance scalablity comes from the amazing GHC compiler and runtime. GHC provides fast code and built-in evented asynchronous IO.

But Yesod is even more focused on scalable development. The key to achieving this is applying Haskell's type-safety to an otherwise traditional MVC REST web framework.

Of course type-safety guarantees against typos or the wrong type in a function. But Yesod cranks this up a notch to guarantee common web application errors won't occur.
\begin{itemize}
\item declarative routing with type-safe urls --- say goodbye to broken links
\item no XSS attacks --- form submissions are automatically sanitized
\item database safety through the Persistent library \cref{persistent} --- no SQL injection and queries are always valid
\item valid template variables with proper template insertion --- variables are known at compile time and treated differently according to their type using the shakesperean templating system.
\end{itemize}

When type safety conflicts with programmer productivity,
Yesod is not afraid to use Haskell's most advanced features
of Template Haskell and quasi-quoting to provide
easier development for its users. In particular, these are used for
declarative routing, declarative schemas, and compile-time templates.

MVC stands for model-view-controller. The preferred library for models
is Persistent~\cref{persistent}. Views can be handled by the Shakespeare family of compile-time template languages. This includes Hamlet, which takes the tedium out of HTML. Both of these libraries are optional, and you can use any Haskell alternative. Controllers are invoked through declarative routing and can return different representations of a resource (html, json, etc).

Yesod is broken up into many smaller projects and leverages Wai \cref{wai} to communicate with the server. This means that many of the powerful features of Yesod can be used in different web development stacks that use WAI such as Scotty.

The new 1.2 release of Yesod, introduces a number of simplifications, especially to the subsite handling. Most applications should be able to upgrade easily. Some of the notable features are:

\begin{itemize}
\item Much more powerful multi-representation support via the selectRep/provideRep API.
\item More efficient session handling.
\item All Handler functions live in a typeclass, providing you with auto-lifting.
\item Type-based caching of responses via the cached function.
\item More sensible subsite handling, switch to HandlerT/WidgetT transformers.
\item Simplified dispatch system, including a lighter-weight Yesod.
\item Simplified streaming data mechanism, for both database and non-database responses.
\item Completely overhauled yesod-test, making it easier to use and providing cleaner integration with hspec.
\item yesod-auth's email plugin now supports logging in via username in addition to email address.
\item Refactored persistent module structure for clarity and ease-of-use.
\item Easy asset combining for static javascript and css files
\item Faster shakespeare template reloading and support for TypeScript templates.
\end{itemize}

Since the 1.0 release, Yesod has maintained a high level of API stability, and we intend to continue this tradition. The 1.2 release introduces a lot of potential code breakage, but most of the required updates should be very straightforward. Future directions for Yesod are now largely driven by community input and patches. We've been making progress on the goal of easier client-side interaction, and have high-level interaction with languages like Fay, TypeScript, and CoffeScript.

The Yesod site (\url{http://www.yesodweb.com/}) is a great place for information. It has code examples, screencasts, the Yesod blog and --- most importantly --- a book on Yesod.

To see an example site with source code available, you can view Haskellers~\cref{haskellers} source code: (\url{https://github.com/snoyberg/haskellers}).

\FurtherReading 
\url{http://www.yesodweb.com/} 
\end{hcarentry} 
